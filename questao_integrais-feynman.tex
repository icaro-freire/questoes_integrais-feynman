\documentclass[11pt, a5paper]{exam}

%==============================================================================
% Pacotes
%------------------------------------------------------------------------------
\usepackage{eulervm}
\usepackage{fontspec}
 \setmainfont
 [
  Path           = fonts/,
  Extension      = .ttf,
  UprightFont    = *-Regular,
  BoldFont       = *-Bold,
  ItalicFont     = *-Italic
 ]{Alegreya}
\usepackage{polyglossia}
  \setdefaultlanguage[variant = brazilian]{portuguese}
\usepackage
 [
	 top    = 2.54cm,
  bottom = 2.54cm,
		left   = 1.91cm,
		right  = 1.91cm
	]{geometry}
\usepackage{coffee4}
\usepackage{graphicx}
	\graphicspath{{./figs/}}
\usepackage{float}
\usepackage[labelfont = bf, font = small]{caption}
\usepackage{fancybox}
\usepackage{booktabs}
\usepackage{multicol}
\usepackage{multirow}
\usepackage{amsmath, amsthm, amsfonts, amssymb, amscd}
\usepackage{mathtools}
\usepackage{systeme}
\usepackage{esint}  
\usepackage{cancel}
\usepackage{esvect}
\usepackage{array}
 \setcounter{MaxMatrixCols}{20}
\usepackage{xcolor}
\usepackage{hyperref} 																											
 \hypersetup
 {
  colorlinks  = true,
  linkcolor   = blue,
  citecolor   = blue,
  filecolor   = blue,
  urlcolor    = blue,
  pdfproducer = {LaTeX},
  pdfcreator  = {XeLaTeX},
  pdfauthor   = {Ícaro Vidal Freire},
  pdfsubject  = {Resposta para questão de Kedna}, 
  pdfkeywords = {Dúvida, Vetorial, Integral Dupla, Matemática, Física,
		               Volume, coordenada polar, coordenadas cilíndricas}
 }
\usepackage{lipsum}
%==============================================================================

%==============================================================================
% Operadores Matemáticos
%------------------------------------------------------------------------------
\DeclareMathOperator{\sen}{sen}
\DeclareMathOperator{\tg}{tg}
\DeclareMathOperator{\cossec}{cossec}
\DeclareMathOperator{\cotg}{cotg}
\DeclareMathOperator{\arcsen}{arcsen}
\DeclareMathOperator{\arctg}{arctg}
\DeclareMathOperator{\arcsec}{arcsec}
\DeclareMathOperator{\Ln}{Ln}
\DeclareMathOperator{\Arg}{Arg}
\DeclareMathOperator{\cis}{cis}
\DeclareMathOperator{\vol}{vol}
%==============================================================================

%==============================================================================
% Novos Comandos 
%------------------------------------------------------------------------------
\newcommand{\dd}{\,\mathrm{d}}
\newcommand{\vazio}{\varnothing}
\newcommand{\Abs}[1]{\vert{#1}\vert}
\newcommand{\versor}[1]{\cdot\vec{\textbf{#1}}}
\newcommand{\Cis}[1]{\cos{#1} + i\,\sen{#1}}
\newcommand{\intc}{\varointctrclockwise}
\newtheorem{teorema}{Teorema}
\newtheorem{obs}{Observação}
%==============================================================================

%==============================================================================
% Classe Exam
%------------------------------------------------------------------------------
\pointpoints{\,ponto}{\,pontos} %---> muda o nome dos pontos: singular e plural,
%                                     respectivamente
\pointformat{}
\qformat{\ovalbox{\bfseries Questão \thequestion}\hfill} %---> o hfill `quebra` 
%                                                              a linha
\renewcommand{\solutiontitle}{\noindent{\bfseries \itshape Resposta:}\enspace}%
\SolutionEmphasis{\small}
\shadedsolutions %-----------------------------------> Mostra sombra na solução
\definecolor{SolutionColor}{rgb}{.95,.95,.95}
\printanswers %--------------------------------------------> Mostra as Soluções
\pagestyle{headandfoot}
\runningfootrule
\firstpagefooter{}{\thepage}{}
\runningfooter{}{\thepage}{}
%==============================================================================

%==============================================================================
% Título
%------------------------------------------------------------------------------
\title{\textbf{Cálculo Fracionário}\\ \textbf{Exercícios}}
\author{Amargosa-BA}
\date{2021.1}
%==============================================================================

%%%%%%%%%%%%%%%%%%%%%%%%%%%%%%%%%%%%%%%%%%%%%%%%%%%%%%%%%%%%%%%%%%%%%%%%%%%%%%%
% INÍCIO DO DOCUMENTO
%------------------------------------------------------------------------------
\begin{document}
%
\maketitle
%
\cofeAm{1}{.7}{94}{7cm}{2.5cm} %---------------> configuração da mancha de café
%
\begin{abstract}
Conjunto de questão sobre a técnica de integração popularizada por Feynman.
As questões foram retiradas do Capítulo VI, do livro \textit{Advanced Calculus}, 
de Frederick S. Woods.
\end{abstract}
%
\begin{questions}
%

% Questão 01 ------------------------------------------------------------------
\question
Se $ f(x) $ é uma função ímpar, isto é, $ f(-x) = -f(x) $, prove que
\[
  \int_{-a}^{a} f(x) \dd{x} = 0. 
\] 
%------------------------------------------------------------------------------

% Questão 02 ------------------------------------------------------------------
\question 
Se $ f(x) $ é uma função par, isto é, $ f(-x) = f(x) $, prove que
\[
  \int_{-a}^{a} f(x) = 2\int_{0}^{a} f(x) \dd x. 
\] 
%------------------------------------------------------------------------------

% Questão 03 ------------------------------------------------------------------
\question 
Se $ f(a - x) = f(x) $, prove que
\[
  \int_{0}^{a} f(x) \dd{x} = 2 \int_{0}^{\frac{1}{2}a} f(x) \dd{x}.
\]
%------------------------------------------------------------------------------

% Questão 04 ------------------------------------------------------------------
\question 
Sendo  $k$ um inteiro positivo, mostre que 
\[
  \int_{0}^{2k\pi} f(\sen{x}) \dd{x} = k \int_{0}^{2\pi}  f(\sen{x}) \dd{x}
\] 
%------------------------------------------------------------------------------

% Questão 05 ------------------------------------------------------------------
\question 
Se $ f(x) $ possui período $ a $, isto é, $ f(x + a) = f(x) $, prove que
\[
  \int_{0}^{ka} f(x) \dd{x} = k \int_{0}^{a} f(x) \dd{x},
\]
onde $ k $ é qualquer inteiro.
%------------------------------------------------------------------------------

% Questão 06 ------------------------------------------------------------------
\question 
Se $ a < b $; e $ f_1(x) < f_2(x) < f_3(x) $ para todo $ x $ no intervalo 
$ (a, b) $, prove que:
\[
  \int_{a}^{b} f_1(x) \dd{x} < \int_{a}^{b} f_2(x) \dd{x} < \int_{a}^{b} f_3(x) \dd{x}.
\]
%------------------------------------------------------------------------------

% Questão 07 ------------------------------------------------------------------
\question 
Se $ m $ e $ M $ são, respectivamente, o \texttt{menor} e o \texttt{maior} valor 
de $ f(x) $ no intervalo $(a, b)$; bem como, $ \phi{(x)} > 0 $, neste mesmo 
intervalo, prove que:
\[
  m \int_{a}^{b} \phi(x) \dd{x} < \int_{a}^{b} f(x) \phi{(x)} \dd{x} < M \int_{a}^{b} \phi{(x)} \dd{x}
\]
e, portanto, para $ a < \xi < b $:
\[
  \int_{a}^{b} f(x) \phi{(x)} \dd{x} = f(\xi) \int_{a}^{b} \phi{(x)} \dd{x}.
\]
%------------------------------------------------------------------------------

% Questão 08 ------------------------------------------------------------------
\question 
Calcule $ \displaystyle \int_{0}^{3} \left( 1 + x^2 \right)^{3/2} \dd{x} $, pela
regra de Simpson, com $ n = 3 $.
%------------------------------------------------------------------------------

% Questão 09 ------------------------------------------------------------------
\question 
Calcule $ \displaystyle \int_{1}^{3} \frac{\dd{x}}{\left( 1 + x^2 \right)^{2}} \dd{x} $, 
pela regra de Simpson, com $ n = 2 $.
%------------------------------------------------------------------------------

% Questão 11 ------------------------------------------------------------------
\question 
Calcule $ \displaystyle \int_{1}^{\pi/3} \log{(\cos{x})}\dd{x} $, 
pela regra de Simpson, com $ n = 2 $.
%------------------------------------------------------------------------------

% Questão 12 ------------------------------------------------------------------
\question 
Examine a integral 
$ \displaystyle \int_{0}^{1} \frac{\dd{x}}{\sqrt{\alpha^2 + x^2}} $ para 
continuidade quando $ \alpha = 0 $.
%------------------------------------------------------------------------------

% Questão 13 ------------------------------------------------------------------
\question 
Examine a integral 
$ \displaystyle \int_{0}^{1} \frac{\dd{x}}{\sqrt{\alpha^2 + x^2}} $ na
vizinhança de $ \alpha = 0 $.

%------------------------------------------------------------------------------

% Questão 14 ------------------------------------------------------------------
\question Encontre as derivadas, com respeito a $ \alpha $, das seguintes 
integrais, sem primeiro integrá-las; e, depois, verifique o resultado integrando 
e derivando em seguida.

\begin{multicols}{2}
 \begin{parts}
  \part $ \displaystyle \int_{0}^{\alpha x} \cos{(x + \alpha)} \dd{x} $
  \part $ \displaystyle \int_{0}^{x} \arcsen{\left(\frac{x}{\alpha}\right)} \dd{x} $
  \part $ \displaystyle \int_{0}^{\sqrt{\alpha}} x \dd{x} $
  \part $ \displaystyle \int_{\alpha^2}^{\alpha^8} (x^2 + \alpha^2) \dd{x} $
 \end{parts}
\end{multicols}
%------------------------------------------------------------------------------

% Questão 15 ------------------------------------------------------------------
\question 
Por diferenciação em relação a $ \alpha $, encontre os valores da seguintes 
integrais:
\begin{multicols}{2}
 \begin{parts}
  \part $ \displaystyle \int_{0}^{\pi} \ln{(1 + \alpha \cos{x})} \dd{x} $;
  \part $ \displaystyle \int_{0}^{1} \frac{x^{\alpha} -1 }{\ln{x}} \dd{x} $
 \end{parts}
\end{multicols}
%------------------------------------------------------------------------------

% Questão 16 ------------------------------------------------------------------
\question 
Por sucessivas diferenciações de 
$ \displaystyle \int_{0}^{1} x^n \dd{x} = \frac{1}{n + 1} $, obtenha:
\[
  \int_{0}^{1} x^n\left( \ln{x} \right)^{m} \dd{x} = (-1)^{m} \frac{m!}{(n + 1)^{m + 1}}
\]
%------------------------------------------------------------------------------

% Questão 17 ------------------------------------------------------------------
\question 
Sabendo-se, para $ \alpha > 1 $, que:
\[
  \int_{0}^{\pi} \frac{\dd{x}}{\alpha - \cos{x}} = \frac{\pi}{\sqrt{\alpha^2 -1}},
\]
mostre que:
\[
  \int_{0}^{\pi} \ln{\frac{b - \cos{x}}{a - \cos{x}}} \dd{x} = \pi \ln{\frac{b + \sqrt{b^2 -1}}{a + \sqrt{a^2 -1}}}.
\]
%------------------------------------------------------------------------------

% Questão 18 ------------------------------------------------------------------
\question 
Teste a convergência das seguintes integrais:
\begin{multicols}{2}
 \begin{parts}
  \part $ \displaystyle \int_{1}^{\infty} \frac{\dd{x}}{x\sqrt{1 + x^2}} $.
  \part $ \displaystyle \int_{0}^{\infty} \frac{\sen^2{x}}{x} \dd{x} $.
  \part $ \displaystyle \int_{a}^{\infty} \frac{x^4 \dd{x}}{(x^2 + a^2)^{5/2}} $.
  \part $ \displaystyle \int_{0}^{\infty} e^{-a^2 x^2} \cos{(bx)} \dd{x} $.
  \part $ \displaystyle \int_{0}^{\infty} e^{^{-x^2 - \frac{a^2}{x^2}}} \dd{x} $.
  \part $ \displaystyle \int_{2}^{\infty} \frac{\dd{x}}{\sqrt{x^3 -1}} $.
 \end{parts}
\end{multicols}
%------------------------------------------------------------------------------

% Questão 19------------------------------------------------------------------
\question 
Prove\footnote{o autor indica um caminho semelhante ao adotado no Exemplo 3, \S 62, do livro} 
a convergência de $ \displaystyle \int_{0}^{\infty} \sen{x^2} \dd{x} $.
%------------------------------------------------------------------------------

% Questão 20 ------------------------------------------------------------------
\question 
Prove a convergência de $ \displaystyle \int_{0}^{\infty} \frac{e^{-ax}\sen{mx}}{x} \dd{x} $.
%------------------------------------------------------------------------------

% Questão 21 ------------------------------------------------------------------
\question 
Prove que as seguintes integrais satisfazem as condições de diferenciabilidade 
em relação a $ \alpha $, sob o sinal da integral:
\begin{multicols}{2}
 \begin{parts}
  \part $ \displaystyle \int_{0}^{\infty} e^{-\alpha x} \dd{x} $.
  \part $ \displaystyle \int_{0}^{\infty} e^{-\alpha x^2} \dd{x} $.
  \part $ \displaystyle \int_{0}^{\infty} e^{-b x^2} \cos{(\alpha x)} \dd{x} $.
  \part $ \displaystyle \int_{0}^{\infty} \frac{\dd{x}}{x^2 + \alpha} \dd{x} $.
 \end{parts}
\end{multicols}
%------------------------------------------------------------------------------

% Questão 22 ------------------------------------------------------------------
\question 
Sabendo-se que $ \int\limits_{0}^{\infty} e^{-\alpha x} \dd{x} = \frac{1}{\alpha} $,
obtenha, por diferenciação,
\[
  \int_{0}^{\infty} x^n e^{-\alpha x} \dd{x} = \frac{n!}{\alpha^{n + 1}}.
\]
%------------------------------------------------------------------------------

% Questão 23 ------------------------------------------------------------------
\question 
Sabendo-se que $ \int\limits_{0}^{\infty} e^{-\alpha x^2} \dd{x} = \frac{1}{2} \sqrt{\frac{\pi}{\alpha}} $,
obtenha, por diferenciação,
\[
  \int_{0}^{\infty} x^{2n} e^{-\alpha x^2} \dd{x} = \frac{\sqrt{\pi}}{2}\frac{1 \cdot 3 \cdots (2n -1)}{2^n\alpha^{n + \frac{1}{2}}}.
\]
%------------------------------------------------------------------------------

% Questão 24 ------------------------------------------------------------------
\question 
De $ \int\limits_{0}^{\infty} \frac{\dd{x}}{x^2 + \alpha} = \frac{\pi}{2\sqrt{\alpha}} $,
obtenha, por diferenciação,
\[
  \int_{0}^{\infty} \frac{\dd{x}}{(x^2 + \alpha)^{n + 1}} = \frac{\pi}{2} \cdot \frac{1 \cdot 3 \cdots (2n -1)}{2\cdot 4 \cdots 2n \alpha^{n + \frac{1}{2}}}.
\]
%------------------------------------------------------------------------------

% Questão 25 ------------------------------------------------------------------
\question 
De $ \int\limits_{0}^{\infty} e^{-\alpha x} \cos{mx} \dd{x} = \frac{\alpha}{\alpha^2 + m^2} $,
obtenha, por integração:
\[
  \int_{0}^{\infty} \frac{e^{-\alpha x} - e^{-\beta x}}{x \sec{mx}} \dd{x} = \frac{1}{2} \ln{\frac{\beta^2 + m^2}{\alpha^2 + m^2}}.
\]
%------------------------------------------------------------------------------

% Questão 26 ------------------------------------------------------------------
\question 
De $ \int\limits_{0}^{\infty} e^{-\alpha x} \sen{mx} \dd{x} = \frac{m}{\alpha^2 + m^2} $,
obtenha, por integração:
\[
  \int_{0}^{\infty} \frac{e^{-\alpha x} - e^{-\beta x}}{x \cossec{mx}} \dd{x} = \arctg{\frac{\beta}{m}} - \arctg{\frac{\alpha}{m}}.
\]
%------------------------------------------------------------------------------

% Questão 27 ------------------------------------------------------------------
\question
De $ \int\limits_{0}^{\infty} e^{-\alpha x} \dd{x} = \frac{1}{\alpha} $,
obtenha, por integração:
\[
  \int_{0}^{\infty} \frac{e^{-ax} - e^{-bx}}{x} \dd{x} = \ln{\frac{b}{a}}.
\]
%------------------------------------------------------------------------------

% Questão 28 ------------------------------------------------------------------
\question 
De $ \int\limits_{0}^{\infty} e^{-\alpha^2 x^2} \dd{x} = \frac{\sqrt{\pi}}{2\alpha} $,
obtenha, por integração:
\[
  \int_{0}^{\infty} \frac{e^{-a^2x^2} - e^{-b^2x^2}}{x^2} \dd{x} = (b - a)\sqrt{\pi}.
\]
%------------------------------------------------------------------------------

% Questão 29 ------------------------------------------------------------------
\question 
Investigue a convergência das seguintes integrais:
\begin{multicols}{2}
 \begin{parts}
  \part $ \displaystyle \int_{0}^{1} (\ln{x})^n \dd{x} $.
  \part $ \displaystyle \int_{0}^{1} \frac{\ln{x}}{1 - x^2} \dd{x} $.
  \part $ \displaystyle \int_{0}^{1} \frac{x}{(1 - x^4)^{1/3}} \dd{x} $.
  \part $ \displaystyle \int_{0}^{\infty} \frac{1}{x^{2/3}(1 + x)} \dd{x} $.
  \part $ \displaystyle \int_{1}^{\infty} \frac{1}{x\sqrt{x^2 - 1}} \dd{x} $.
  \part $ \displaystyle \int_{0}^{1} \sqrt{\frac{1 - k^2x^2}{1 - x^2}} \dd{x} $.
 \end{parts}
\end{multicols}
%------------------------------------------------------------------------------

% Questão 30 ------------------------------------------------------------------
\question 
Calcule as integrais que seguem.
Em alguns casos, apenas uma mudança de variável é necessária.

\begin{parts}
 \part $ \displaystyle \int_{0}^{\infty} \frac{\sen{mx}}{x} \dd{x} = 
 \begin{cases} 
 \pi/2,  &\text{ se } m > 0,\\
 0,      &\text{ se } m = 0,\\
 -\pi/2, &\text{ se } m < 0.
 \end{cases}$
 \part $ \displaystyle \int_{0}^{\pi} x \ln{(\sen{x})} \dd{x} = -\frac{\pi^2}{2} \ln{2} $.
 \part $ \displaystyle \int_{0}^{\infty} e^{-\alpha^2 x^2} \dd{x} = \frac{1}{2\alpha} \sqrt{\pi} $.
 \part $ \displaystyle \int_{0}^{1} \frac{\dd{x}}{\sqrt{\ln\frac{1}{x}}} = \sqrt{\pi} $.
 \part $ \displaystyle \int_{0}^{\infty} \frac{\sen{x}\cos{mx}}{x} \dd{x} = 
 \begin{cases} 
 0,     &\text{ se } m < -1 \text{ ou } m > 1, \\
 \pi/4, &\text{ se } m = -1 \text{ ou } m = 1, \\
 \pi/2, &\text{ se } -1 < m < 1.
 \end{cases}$
 \part $ \displaystyle \int_{0}^{\infty} e^{-x^2 - \frac{\alpha^2}{x^2}} \dd{x} = \frac{e^{-2\alpha}\sqrt{\pi}}{2} $.\\
 {\footnotesize \textbf{(Dica):} Representando a integral por $u$, primeiro mostre que $ \frac{\dd{u}}{\dd{a}} = -2u $}
 \part $ \displaystyle \int_{0}^{\infty} \frac{e^{-ax} \sen{bx}}{x} \dd{x} = \arctg{\frac{b}{a}} $.
 \part $ \displaystyle \int_{0}^{\infty} \frac{\cos{x}}{\sqrt{x}} \dd{x} = \int_{0}^{\infty} \frac{\sen{x}}{\sqrt{x}} \dd{x}  = \sqrt{\frac{\pi}{2}} $.
 \part $ \displaystyle \int_{0}^{\pi} \frac{\ln{(1 + k \cos{x})}}{\cos{x}} \dd{x} = \pi \arcsen{k}$, com $ 0 < k < 1 $.
 \part $ \displaystyle \int_{0}^{\pi/2} \ln{\left(\frac{1 + k \sen{x}}{1 - k \sen{x}}\right)} \cdot \frac{\dd{x}}{\sen{x}} = \pi \arcsen{k}$, com $ 0 < k < 1 $.
 \part $ \displaystyle \int_{0}^{\infty} xe^{-\alpha x} \cos{\beta x} \dd{x} = \frac{\alpha^2 - \beta^2}{(\alpha^2 + \beta^2)^2}$.
 \part $ \displaystyle \int_{0}^{a} \sqrt{a^2 - x^2} \arccos\left(\frac{x}{a}\right) \dd{x} = a^2\left(\frac{\pi^2}{16} + \frac{1}{4}\right) $.
 \part $ \displaystyle \int_{0}^{\pi} \frac{\sen{x} \dd{x}}{\sqrt{1 - 2 \alpha \cos{x} + \alpha^2}} = 
 \begin{cases}
 2, &\text{ se } \alpha^2 \leq 1,\\
 \frac{2}{\alpha}, &\text{ se } \alpha > 1.
 \end{cases}$
\end{parts}
%------------------------------------------------------------------------------

% Questão 31 ------------------------------------------------------------------
\question 
Mostre para $ x $ suficientemente grande que:
\[
  \int_{x}^{\infty} e^{-x} \frac{\dd{x}}{x} = e^{-x} \left(\frac{1}{x} - \frac{1}{x^2} + \frac{2!}{x^3} - \ldots + (-1)^n \frac{(n - 1)!}{x^n}\right) + R_n
\]
\begin{parts}
 \part Mostre que a série diverge, mas que $ R_n $ é menor, em valor absoluto, do que o último termo entre parênteses. 
  Esta é uma série assintótica.
\end{parts}

%------------------------------------------------------------------------------

% Questão 32 ------------------------------------------------------------------
\question 
Mostre que
{
\small
 \[
  \int_{x}^{\infty} e^{-x^2} \dd{x} = 
  \frac{e^{-x^2}}{2x}
  \left(\, 
  1 - \frac{1}{2x^2} + \frac{1\cdot 3}{2^2x^4} - \ldots + (-1)^n \frac{1\cdot 3\cdot 5 \cdots (2n -1)}{2^n x^{2n}}
  \,\right) + R_n.
 \]
}
\begin{parts}
 \part Mostre, também, que esta é uma série assintótica, como na questão anterior.
\end{parts}
%------------------------------------------------------------------------------

\end{questions}

%
\end{document}
%------------------------------------------------------------------------------
% FIM DO DOCUMENTO
%%%%%%%%%%%%%%%%%%%%%%%%%%%%%%%%%%%%%%%%%%%%%%%%%%%%%%%%%%%%%%%%%%%%%%%%%%%%%%%